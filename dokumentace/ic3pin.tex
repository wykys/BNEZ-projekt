% We use tikz to construct a shape corresponding to a 3 pin IC
%
% Source: http://www.elfsoft2000.com/projects/multipole.pdf


\newlength{\side}               % We define a new length that corresponds to the side of the IC we are constructing
\setlength{\side}{2cm}

\pgfdeclareshape{ic3pin}
{
   \anchor{center}{\pgfpointorigin}      % All circuitikz objects must have a 'center' and a 'text' anchor. Within the object/node (0, 0) is at the center and everything is drawn around this
   \anchor{text}                        % Used to center the text inside the node
      {\pgfpoint{-.5\wd\pgfnodeparttextbox}{-.5\ht\pgfnodeparttextbox}}

   % Construct custom anchors
   \savedanchor\icpina{\pgfpoint{-0.5\side}{0}}        % Create an anchor with internal name \icpina for pin 1 at the specified point which is on the top of the IC at its center hence the coordinates (0, 0.5\side)
   \anchor{pin1}{\icpina}                               % We create an anchor with external name pin1 from the saved anchor created above

   \savedanchor\icpinb{\pgfpoint{0}{-0.5\side}}
   \anchor{pin2}{\icpinb}

   \savedanchor\icpinc{\pgfpoint{0.5\side}{0}}
   \anchor{pin3}{\icpinc}

   % Now we draw the actual node/object
   \foregroundpath              % Border and pin numbers are drawn here
   {
      \pgfsetlinewidth{0.05cm}

      \pgfpathrectanglecorners{\pgfpoint{-1cm}{-1cm}}{\pgfpoint{1cm}{1cm}}
      \pgfusepath{draw}     % Draw the rectangle defined above
   }
}




\pgfdeclareshape{LM25085}{
\anchor{center}{\pgfpointorigin} % within the node, (0,0) is the center
\anchor{text} % this is used to center the text in the node
{\pgfpoint{-.5\wd\pgfnodeparttextbox}{-.5\ht\pgfnodeparttextbox}}
\savedanchor\icpina{\pgfpoint{-.75cm}{-.625cm}} % pin 1
\anchor{p1}{\icpina}
\savedanchor\icpinb{\pgfpoint{-.25cm}{-.625cm}} % pin 2
\anchor{p2}{\icpinb}
\savedanchor\icpinc{\pgfpoint{.25cm}{-.625cm}} % pin 3
\anchor{p3}{\icpinc}
\savedanchor\icpind{\pgfpoint{.75cm}{-.625cm}} % pin 4
\anchor{p4}{\icpind}
\savedanchor\icpine{\pgfpoint{.75cm}{.625cm}} % pin 5
\anchor{p5}{\icpine}
\savedanchor\icpinf{\pgfpoint{.25cm}{.625cm}} % pin 6
\anchor{p6}{\icpinf}
\savedanchor\icping{\pgfpoint{-.25cm}{.625cm}} % pin 7
\anchor{p7}{\icping}
\savedanchor\icpinh{\pgfpoint{-.75cm}{.625cm}} % pin 8
\anchor{p8}{\icpinh}
\foregroundpath{ % border and pin numbers are drawn here
\pgfsetlinewidth{0.05cm}
\pgfpathrectanglecorners{\pgfpoint{2cm}{4cm}}{\pgfpoint{-2cm}{-4cm}}
\pgfusepath{draw} %draw rectangle


\pgfusepath{draw} %draw semicircle

\pgftext[left,at={\pgfpoint{-1.8cm}{3cm}}]{\scriptsize VIN}
\pgftext[left,at={\pgfpoint{-1.8cm}{1cm}}]{\scriptsize RT}
\pgftext[left,at={\pgfpoint{-1.8cm}{-3cm}}]{\scriptsize GND}

\pgftext[right,at={\pgfpoint{1.8cm}{3cm}}]{\scriptsize VCC}
\pgftext[right,at={\pgfpoint{1.8cm}{1cm}}]{\scriptsize ADJ}
\pgftext[right,at={\pgfpoint{1.8cm}{-1cm}}]{\scriptsize ISEN}
\pgftext[right,at={\pgfpoint{1.8cm}{-2cm}}]{\scriptsize PGATE}
\pgftext[right,at={\pgfpoint{1.8cm}{-3cm}}]{\scriptsize FB}

}}
